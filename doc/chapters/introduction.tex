%
% introduction.tex
%
% Copyright (C) 2021 by SpaceLab.
%
% TTC Documentation
%
% This work is licensed under the Creative Commons Attribution-ShareAlike 4.0
% International License. To view a copy of this license,
% visit http://creativecommons.org/licenses/by-sa/4.0/.
%

%
% \brief Introduction chapter.
%
% \author Gabriel Mariano Marcelino <gabriel.mm8@gmail.com>
%
% \institution Universidade Federal de Santa Catarina (UFSC)
%
% \version 1.2.0
%
% \date 2021/01/16
%

\chapter{Introduction} \label{ch:introduction}

This document is the general documentation of the Telemetry, Tracking and Command module of the FloripaSat project.

The TTC\nomenclature{\textbf{TTC}}{Telemetry, Tracking and Command.} (or TT\&C) is the communication module of the FloripaSat\cite{site} cubesat. It is responsible to make the communication between the earth (A ground station) and the satellite, and is divided in two sub-modules: Beacon and telemetry.

The beacon is a independent sub-module who transmits a periodic signal containing an identification data (ID) of the satellite and some basic telemetry data.

The telemetry sub-module is the main communication device. It has a bidirectional data link to receive telecommands from the earth and transmit all the requested data.

The telemetry sub-module is controlled by the OBDH\nomenclature{\textbf{OBDH}}{Onboard Data Handling.} module (The OBC\nomenclature{\textbf{OBC}}{Onboard Computer.} of the satellite), and the software for handling with this sub-module is under development in the OBDH module, and is not documented here.

\nomenclature{\textbf{PCB}}{Printed Circuit Board.}
\nomenclature{\textbf{USB}}{Universal Serial Bus.}

\cite{github}

The beacon executes the following tasks during its execution:

\begin{itemize}
	\item 145 MHz band antenna deployment.
	\item One-way communication (RX) with the EPS module, using the FSP protocol.
	\item Two-way communication (TX and RX) with the OBDH module, using the FSP protocol.
	\item Transmission of the beacon packets (In two protocols: NGHam and AX.25), containing the data from the EPS or the OBDH module and the satellite ID (``FLORIPASAT").
	\item When required by the OBDH module, the transmissions are stopped for an hibernation period (shutdown).
	\item In case of an critical failure of the OBDH module, or the uplink channel, the beacon activates its reception and is capable of make its own hibernation (shutdown).
\end{itemize}

\section{Module Requirements}

In the list below, the TTC module requirements for the mission are described. These requirements are nominated as TMR\nomenclature{\textbf{TMR}}{Telemetry, Tracking and Command Module Requirements.}, or Telemetry, Tracking and Command Module Requirements.

\begin{enumerate}[label=\textit{TMR \arabic*}, leftmargin=*, align=left]
	\item The FloripaSat shall have a physical device to inhibit radio frequency (RF\nomenclature{\textbf{RF}}{Radio Frequency.}) transmission.
	
	\begin{footnotesize}
		Compliance with CDS\nomenclature{\textbf{CDS}}{CubeSat Design Specification.} 3.3.9: The use of three independent inhibits is highly recommended and can reduce required documentation and analysis.	
	\end{footnotesize}
	
	\item The CubeSat will have the RF power output to the transmitting antenna input no greater than 1,5 W.
	
	\begin{footnotesize}
		Compliance with CDS 3.3.9.1.
	\end{footnotesize}
	
	\item The CubeSat will have the RF power output to the transmitting antenna input no less than 1,0 W (or 30 dBm).
	
	\begin{footnotesize}
		Defined by team analysis.
	\end{footnotesize}
	\item No CubeSats shall generate or transmit any RF signal from the time of integration into the P-POD\nomenclature{\textbf{P-POD}}{Poly-Picosatellite Orbital Deployer.} through 45 minutes after on-orbit deployment from the P-POD.
	
	\begin{footnotesize}
		Compliance with CDS.
	\end{footnotesize}
	
	\item TTC transceiver shall transmit and receive on the frequency of 437,9 Mhz.
	
	\begin{footnotesize}
		Defined by the team, based on available spectrum allocation to Amateur communication.
	\end{footnotesize}
	
	\item TTC beacon shall transmit on the frequency of 145,9 Mhz.
	
	\begin{footnotesize}
		Defined by the team, based on available spectrum allocation to Amateur communication.
	\end{footnotesize}
	
	\item TTC shall modulate and demodulate information using GFSK\nomenclature{\textbf{GFSK}}{Gaussian Frequency-Shift Keyring.}.
	
	\begin{footnotesize}
		Defined by the team.
	\end{footnotesize}
	
	\item TTC Beacon must transmit periodic beacon messages at an interval of 10
seconds, except when in hibernation or shutdown mode. Allows ground stations to track and receive satellite data even if telecommand was not sent to the satellite.
	\item TTC transceiver must receive signals from ground stations and demodulate them.
	\item TTC must interface with OBDH, exchanging encoded raw data received or to be transmitted.
	\item TTC must interface with OBDH using the SPI protocol ($@$2 KHz).
	
	\begin{footnotesize}
		Defined by the team.
	\end{footnotesize}
	
	\item TTC radio must modulate raw data received from OBDH using GFSK, prior to transmission, and demodulate received data and forward raw data to OBDH.
	\item TTC transceiver shall transmit and receive data at a baud rate of 2400 bps. Defined by the team, based on link budget analysis.
	\item TTC beacon shall transmit data at a baud rate of 1200 bps.
	
	\begin{footnotesize}
		Defined by the team, based on link budget analysis.
	\end{footnotesize}
	
	\item TTC beacon shall transmit packets using the NGHam and AX.25 protocols.
	
	\begin{footnotesize}
		Defined by the team.
	\end{footnotesize}
	
	\item A same beacon packet must be transmitted in both NGHam and AX.25 protocols.
	
	\begin{footnotesize}
		Defined by the team.
	\end{footnotesize}
	
	\item TTC must receive the batteries voltages from the EPS module at every 10 seconds.
	
	\begin{footnotesize}
		Defined by the team.
	\end{footnotesize}
	
	\item The payload from the packets transmitted by the beacon must contain at least the satellite ID ("FLORIPASAT") and batteries voltages (received from the EPS module at every 10 seconds).
	\item TTC uC must perform the antenna deployment of the VHF band antenna.
	
	\begin{footnotesize}
		Defined by the team.
	\end{footnotesize}
	
	\item Between the beacon packets transmissions, the beacon MCU and radio must operate in low power mode, to save energy.
	
	\begin{footnotesize}
		Defined by the team.
	\end{footnotesize}
	
	\item TTC PAs (Power amplifiers) must just only be activated during transmissions. When they are not in operation, they must be turned off.
	
	\begin{footnotesize}
		Defined by the team.
	\end{footnotesize}
	
	\item All the beacon critical data, like time control and antenna deployment status, must be stored in a non-volatile memory.
	
	\begin{footnotesize}
		Defined by the team.
	\end{footnotesize}
	
	\item TTC beacon must be able to receive a 24 hour shutdown command from the OBDH module.
	
	\begin{footnotesize}
		Compliance with AMSAT/IARU regulations.
	\end{footnotesize}
	
\end{enumerate}
